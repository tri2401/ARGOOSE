Possible future implementations may, if possible, include advancements in the current system specifications. The system does not address historical data factors and may be included in future updates. As well as this, the ability to real time track, and also to track in real time multiple drones will not be included in the base model design. Altitude detection may, depending on time, be included but is restricted to a future implementation as it is a possible blocker for other core implementations at this time. Tertiary detection methods may also be added after initial system is completed and expandability is addressed.

\subsection{System will track multiple drones at one time}
\subsubsection{Description}
The system will be capable of selecting and actively tracking multiple drones it detects at any given time.
\subsubsection{Source}
Future Requirement
\subsubsection{Constraints}
Degraded performance due to weather effects may impact performance.
\subsubsection{Standards}
There are no intrinsic standards to meet this requirement.
\subsubsection{Priority}
This is a possible Future implementation.

\subsection{System will provide real time response to multiple drone scenarios}
\subsubsection{Description}
While actively tracking multiple drones, the system will provide close to real time data updates to be displayed.
\subsubsection{Source}
Future Requirement
\subsubsection{Constraints}
Degraded performance due to weather effects may impact performance.
\subsubsection{Standards}
There are no intrinsic standards to meet this requirement.
\subsubsection{Priority}
This is a possible Future implementation.

\subsection{The system shall be able to detect altitude of UAVs}
\subsubsection{Description}
On top of determining the top-down location of drones, the system should also be able to determine the height of where the drones are.  The technique for this has yet to be determined, but the output shall simply be a number and metric displayed somewhere on the user interface (e.g., 70 meters, 120 ft); more than likely, this feature will only be implemented for one drone at a time.
\subsubsection{Source}
Augustine Nguyen
\subsubsection{Constraints}
Altitude calculation will be for drones within a set area (yet to be determined), most definitely within the radius of the main detection system.  It will also be done for one drone at time.
\subsubsection{Standards}
N/A
\subsubsection{Priority}
Future

\subsection{The system shall send an email for the alert if a drone passes by the perimeter.}
\subsubsection{Description}
The user will have choice to select the medium of alert as email or text. Alerts are configurable and can also be sent to local law enforcement for a coordinated response.
\subsubsection{Source}
Sanyogita Piya
\subsubsection{Constraints}
N/A
\subsubsection{Standards}
N/A
\subsubsection{Priority}
Future

\subsection{Third Detection Layer: Computer vision implementation}
\subsubsection{Description}
If time and budget allows for it, detection through visual data stream can add another layer of drone detection verification. Although it seems unviable at the moment, a future implementation would add to the system's functionality.
\subsubsection{Source}
Mahin Roddur
\subsubsection{Constraints}
Budgetary and time constraints are the most apparent one here. 
\subsubsection{Standards}
N/A
\subsubsection{Priority}
Future
