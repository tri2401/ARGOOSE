This section outlines how customers should expect a simple and easy to use drone detection system.  It should be able to detect drones within a certain radius and display information for the user to view.  The look and feel is like google maps with drone detection.

\subsection{System will accept verification of false positives by user}
\subsubsection{Description}
Utilizing the web interface, whenever a positive drone detection has been emitted by the system: The user will be able to manually identify that the drone is not present where detected and inform the system that it is a false positive result. After the system acknowledges this change, it will no longer track the false positive result and store an identifier so that rates of false positives can be assessed by the end user.
\subsubsection{Source}
Customer Requirement
\subsubsection{Constraints}
There are no realistic constraints to this requirement. Utilizing the web interface to remove detections and storing results should be accomplished without incident.
\subsubsection{Standards}
There are no intrinsic standards to meet this requirement.
\subsubsection{Priority}
This is a priority Low requirement.

\subsection{System will be able to actively track an individual drone}
\subsubsection{Description}
Utilizing the web interface, when the system has positively identified a drone, the user will be able to select that drone. Upon selection, the tracking system will switch to an active tracking mode and attempt to track the selected drone in as close to real time as possible.
\subsubsection{Source}
Customer Requirement
\subsubsection{Constraints}
Degraded performance due to weather effects may impact performance.
\subsubsection{Standards}
There are no intrinsic standards to meet this requirement.
\subsubsection{Priority}
This is a Critical requirement.

\subsection{System will detect multiple drones with passive scanning}
\subsubsection{Description}
Utilizing the web interface, the system when not actively tracking an individual drone will passively scan the area. During these scans it will return sweep ping results of potential drones it detects with the acoustic + subsystem algorithm.
\subsubsection{Source}
Customer Requirement
\subsubsection{Constraints}
Degraded performance due to weather effects may impact performance.
\subsubsection{Standards}
There are no intrinsic standards to meet this requirement.
\subsubsection{Priority}
This is a Critical requirement.

\subsection{Product shall be delivered by the end of May 2023 school year}
\subsubsection{Description}
The product must be in a minimum viable state by the end of May 2023 school year, in time to present and demonstrate.  Minimum viable state means a state in which the product can reach at least 60\% of its main functionalities as described in the product description, performance requirements, and other customer requirements, on top of having ALL the packaging requirements.
\subsubsection{Source}
Augustine Nguyen
\subsubsection{Constraints}
The biggest constraint is the document itself.
\subsubsection{Standards}
UTA academic codes
\subsubsection{Priority}
Critical

\subsection{The system shall be developed within a production budget of \$800}
\subsubsection{Description}
Since this system is being developed as a Senior Design project funded by UTA, there is a budget constraint that can factor into functionality and long-term maintenance limitations.
\subsubsection{Source}
CSE Senior Design Specifications mention this general budgetary constraint on most projects. 
\subsubsection{Constraints}
Exceptions require permission from the CSE department on a case-by-case basis.
\subsubsection{Standards}
N/A
\subsubsection{Priority}
Critical
