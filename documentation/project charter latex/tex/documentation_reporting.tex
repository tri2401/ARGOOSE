%%% In this section, you will describe all of the various artifacts that you will generate and maintain during the project life cycle. Describe the purpose of each item below, how the content will be generated, where it will be stored, how often it will be updated, etc. Replace the default text for each section with your own description. Reword this paragraph as appropriate.

\subsection{Major Documentation Deliverables}
The major deliverables are this project charter, system requirements specification, architectural design specification, and detailed design specification. These are the main documents that will be turned in during the first half of the project. Each of these document will be prepared with the co-operation of the whole team and stored in team's github as we move forward with the project. These documents will also be updated as needed during the course of the project. 

\subsubsection{Project Charter}
This document will be maintained as a team and updated during each sprint if any requirement is edited, added or removed. The initial version will be delivered on October 3rd, 2022 and the final version will be delivered along with the complete project on May of 2023.

\subsubsection{System Requirements Specification}
This document will be maintained as a team and updated as needed. The requirements will not be edited as much as other documents, but it will be updated if the team decides to add or remove some parts. The initial version will be delivered towards the end of October 2022 and the final version will be delivered along with the complete project on May of 2023.

\subsubsection{Architectural Design Specification}
This document will be maintained as a team and updated frequently during the first half of the project. In the second half as we progress in the project and we have a solid structure for the project, it will be updated few times. The initial version will be delivered on November of 2022 and the final version will be delivered along with the complete project on May of 2023.

\subsubsection{Detailed Design Specification}
This document will be maintained as a team and updated frequently during the initial stages. However, it will not be updated much after the completion of the design of the project. If a major change happens in the project, it will be updated accordingly. The initial version will be delivered on October 3rd, 2022 and the final version will be delivered along with the complete project on May of 2023.

\subsection{Recurring Sprint Items}
\subsubsection{Product Backlog}
The items will be added after the discussion with the whole team after splitting up tasks so that no item on the backlog is too big. These items will be prioritized according to the need for the software. The decisions will be based on the majority vote. The software to maintain and share the product backlog with team members and stakeholders will be Jira.

\subsubsection{Sprint Planning}
Each sprint will be planned before the beginning of the sprint. The backlog can be updated during the sprint. In Senior Design 1, there will be 4 sprints. In Senior Design 2, there will also be 4 sprints.

\subsubsection{Sprint Goal}
The Sprint goal is decided by the group together with some feedback from the sponsor and customers of the project. 

\subsubsection{Sprint Backlog}
The team together will decide the backlog items and they will be kept in the Jira Board by the scrum master. The backlog will be maintained by the whole team as they progress in their allocated tasks with options like in-progress, done, etc.

\subsubsection{Task Breakdown}
Individual task will be assigned by group discussion and the preference of each member. If there is a conflict, then the group leader for the sprint will decide how the conflicted task will be divided. Time spent on tasks will be documented using man hour.

\subsubsection{Sprint Burn Down Charts}
The scrum master will be responsible for generating the burn down charts for each sprint. The Jira Board has a feature to keep track of time spent on each task and Jira Board will create a burn down chart accordingly.

\begin{figure}[h!]
    \centering
    \includegraphics[width=0.5\textwidth]{images/test_image}
    \caption{Example sprint burn down chart}
\end{figure}

\subsubsection{Sprint Retrospective}
How will the sprint retrospective be handled as a team? When will this discussion happen after each sprint? What will be documented as a group and as individuals, and when will it be due?
The Sprint retrospective will be handled after the day of completion of each sprint. As a group, problems faced, lessons learned, ways to improve communication and the allocation of task will be documented. As individuals, lessons learned and peer reviews will be documented. These are due according to the schedule provided by the professor.

\subsubsection{Individual Status Reports}
In the progress discussion meeting, each member will discuss the progress made by them and the problems they faced. The key items that will be included in the report will be which items they have worked on, problems they faced, if they need any assistance, etc.

\subsubsection{Engineering Notebooks}
The engineering notebook will be updated after a change is brought up in the project. At a minimum, it will be updated bi-weekly which is the length of each sprint. There will be no minimum amount of pages. Team members will be each others' witness and will keep each other accountable.

\subsection{Closeout Materials}
\subsubsection{System Prototype}
What will be included in the final system prototype? How and when will this be demonstrated? Will there be a Prototype Acceptance Test (PAT) with your customer? Will anything be demonstrated off-site? If so, will there be a Field Acceptance Test (FAT)?
The system prototype will include....................

\subsubsection{Project Poster}
Currently, this has not been discussed and will be done in the later part of the project.

\subsubsection{Web Page}
What will be included on the project web page? Will it be accessible to the public? When will this be delivered? Will it be updated throughout the project, or just provided at closeout (at a minimum, you need to provide a simple web page at the end).
The project web page will include 

\subsubsection{Demo Video}
What will be shown in the demo video(s)? Will you include a B-reel footage for future video cuts? Approximately how long will the video(s) be, and what topics will be covered?

\subsubsection{Source Code}
How will your source code be maintained? What version control system will you adopt? Will source code be provided to the customer, or binaries only? If source code is provided, how will it be turned over to the customer? Will the project be open sourced to the general public? If so, what are the license terms (GNU, GPL, MIT, etc.). Where will the license terms be listed (in each source file, in a single readme file, etc.).

\subsubsection{Source Code Documentation}
What documentation standards will be employed? Will you use tools to generate the documentation (Doxygen, Javadocs, etc.). In what format will the final documentation be provided (PDF, browsable HTML, etc.)?

\subsubsection{Hardware Schematics}
Will you be creating printed circuit boards (PCBs) or wiring components together? If so, list each applicable schematic and what sort of data it will contain (PCB layout, wiring diagram, etc.). If your project is purely software, omit this section.

\subsubsection{CAD files}
Will the project involve any mechanical design, such as 3D printed or laser-cut parts? If so, what software will you use to generate the files and what file formats will you provide in your closeout materials (STL, STEP, OBJ, etc.). If your project is purely software, omit this section.

\subsubsection{Installation Scripts}
How will the customer deploy software to new installations? Will you provide installation scripts, install programs, or any other tools to improve the process? Will there be multiple scripts provided (perhaps separate scripts for the graphical front end and back end server software)? 

\subsubsection{User Manual}
Will you customer need a printed or digital user manual? Will they need a setup video? Decide now what will be provided and discuss.
