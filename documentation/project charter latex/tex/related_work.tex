Unmanned aerial vehicles have evolved rapidly over the past few decades leading to mass production of affordable drones so has its risks therefore there are many counter measures research done in the detection and tracking of the drones. Some are mentioned below: 
\begin{itemize}
  \item  \textbf{Drone Police Department}\cite{Nguyen2017}: Project by University of Colorado funded by Department of Homeland Security. The project explores the feasibility of inexpensive RF-based detection of the presence of drones and examines whether physical characteristics of the drone, such as vibration and shifting, can be detected in the wireless signal transmitted. The downside of this project is that it works for seven different types of drones which emit radio frequency and the drones which do not emit radio frequency will not be detected by this software.
  
  \item \textbf{Automated Drone Detection Using YOLOv4}\cite{Singha2021}: A research paper published by the Department of Computer Science and Information Systems of Texas A&M University which designed an automated drone detection system using YOLOv4. YOLO is a one-stage object detection model. The model is trained using drone and bird data-sets acquired from the public data-sets. Detecting drones at various altitudes are difficult due to their small size and speed as well as the existence of drone-like objects.
  \item \textbf{Eye in the Sky}\cite{Pratt2015}:  Drone Detection & Tracking System:  It is an Airport Cooperative Research Program introduced in the University Design Competition for Addressing Airport Needs conducted by the University of Rhode Island. There are two project components to this project, one is to provide the airport the ability to detect when a drone enters the critical airspace surrounding an airport and to inform the operator if the drone is within the 5 mile safety buffer around an airport.  It uses GigE Vision Camera with MATLAB to implement it. The cost aspect of this project made it difficult to be relevant to our project.
  \item \textbf{EchoGuard}\cite{EchoGuard}:  It is a product from EchoDyne which uses ultra-low SWaP ESA radar that detects and tracks drones in unauthorized areas. It rapidly and accurately slews other sensors such as PTZ optical cameras for continuous eyes-on-object, even at high zoom levels and while tracking fast moving targets. Radars are expensive and this uses a military radar which are seriously expensive.
  \item \textbf{Drone Detection and Defense Systems}\cite{DroneDetectionChiper}: The research supported by the Romanian Ministry of Education and Research. The research describes their our own solution that was designed and implemented in the framework of the DronEnd research project. The DronEnd system is based on RF methods and uses SDR platforms as the main hardware elements. 
\end{itemize}
