Since the 9/11 incident, the United States have been seen using military-grade drones as an offensive weapon against enemies of the state \cite{yoo2011assassination}.  Five years later, the Federal Aviation Administration started issuing permits for the use of drone in a consumer setting; this opened possibilities for the use of drones, whether it be for videography, entertainment, or utility purposes.  At the same time, this posed a threat to aviation security:  a drone flying around an airport can cause a number of safety issues, due to how precise pilots have to be; drones can also be used as weapons, strapping explosive devices to them and flying them under radar; being quiet and low-profile, an additional use for drones can be espionage or spying \cite{yaacoub2020security}.  All of these hazardous uses for drones are reasons why a drone detection system would be beneficial, whether it be for airport, military, or home security \cite{kardasz2016drones}.  Currently, radar is the main mode of drone detection; however, radar by itself isn't always enough.  First off, the type of radar would have to be of high-resolution, but even then, the radar wouldn't be able to distinguish the drone from other smaller flying objects like birds \cite{coluccia2020detection}.  High-resolution radar would have to be supplemented with databases of stored signatures, alongside AI and machine-learning software to be able to completely distinguish drones from other flying entities.  The sponsor of this project is Christopher D. McMurrough, a professor at the University of Texas at Arlington--our instructor for senior design.  The intention for this project is to find potential customers by developing an attractive and alternative solution to drone detection; the applications of a light-weight and cost-effective drone detection system can prove to be both helpful and lucrative.